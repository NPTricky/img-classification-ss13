\documentclass[liststotoc,11pt,a4paper]{article}

\usepackage[scaled=0.88]{beraserif}
\usepackage[scaled=0.85]{berasans}
\usepackage[scaled=0.84]{beramono}

\usepackage{mathtext}
\usepackage[T1]{fontenc}            
\usepackage[utf8]{inputenc}
\usepackage[english,ngerman]{babel}
\usepackage{amsmath,amssymb,amsthm}
\usepackage{mathtools}
\usepackage{isomath}

\usepackage{mathpazo}
\linespread{1.05}
\usepackage[T1,small,euler-digits]{eulervm}

\usepackage{layouts}
\usepackage[pdftex]{graphicx}

\usepackage[
twoside, % zweiseitiger Druck
left=2.5cm, % linker Rand
right=2.5cm, % rechter Rand
top=1.5cm, % oberer Rand
bottom=2.5cm, % unterer Rand
]{geometry}
\usepackage[printonlyused]{acronym} %Abkürzungen
\usepackage{float}
\usepackage{capt-of}

\begin{document}
% ------------------------------------------------- Deckblatt -------------------------------------------------------------
\begin{figure}[htbp]
\begin{minipage}[t]{5cm}
\vspace{0pt}
\includegraphics{ovgu_logo_fak_inf}
\end{minipage}
\hfill
\begin{minipage}[t]{8cm}
\vspace{0pt}
\begin{flushright}     
Otto-von-Guericke-Universität Magdeburg\\Fakultät für Informatik\\AG Bildverarbeitung und Bildverstehen\\
\end{flushright}
\end{minipage}
\end{figure}
\vspace{50pt}

\begin{center}
\huge\bfseries Report zu Projekt Samael\\
\vspace{10pt}
\large"`The Blind God"'\\
\end{center}
\begin{center}
\vspace{20pt}
------------------------------------------------------------------------------------------------------------------\\[30pt]
\large Klassifikation von Bildern\\
\vspace{10pt}
\normalsize 10.07.2013 \\[220pt] 
\end {center}

\begin{figure}[htbp]
\begin{minipage}[t]{4.5cm}
\begin{flushleft}
Tim Benedict Jagla \\
Computervisualistik\\
Matrikelnr.: 187768\\ 
tim.jagla@st.ovgu.de\\
\end{flushleft}
\end{minipage}
%\hfill
\begin{minipage}[t]{4.8cm}
\begin{flushleft}
Christoph Lämmerhirt\\
Informatik\\
Matrikelnr.: ...\\
christoph.laemmerhirt@st.ovgu.de\\
\end{flushleft}
\end{minipage}
\hfill
\begin{minipage}[t]{4.5cm}
\begin{flushleft} 
Sarah Pauksch\\
Computervisualistik\\
Matrikelnr: 188145\\ 
sarah.pauksch@st.ovgu.de\\
\end{flushleft}
\end{minipage}
\end{figure}

\vspace{\fill}

\begin{figure}[htbp]
\begin{minipage}[b]{0.475\textwidth}
\vspace{0pt}
\begin{flushright}    
\begin {tabbing}
\hspace*{5cm}\=\hspace{2,5cm}\=\hspace{5cm}\=\hspace{2.5cm}\=\hspace{2.5cm}\=\kill
Professor:	\>Prof. Dr.-Ing. Klaus Tönnies\\
Kurs: \>Advanced Topics in Image Understanding\\

\end{tabbing}
\end{flushright}
\end{minipage}
\hfill
\begin{minipage}[b]{0.475\textwidth}
\centering
\end{minipage}
\end{figure}


\thispagestyle{empty}
\newpage
\setcounter{page}{1}
\pagenumbering{arabic}
%-----------------------------------------Verzeichnisse-----------------------------------------------------------------
%------------------------------                                   ------------------------------------------------------
\tableofcontents                % automatisches Inhaltsverzeichnis
\newpage

%------------------------------------------------ eigentlicher Text -----------------------------------------------------

% ---------------------------------------------------- Einleitung ----------------------------------------------------- %
\Large \bfseries Abstrakt\\
\normalsize \mdseries
\addcontentsline{toc}{section}{Abstrakt}
\\Im Rahmen von Projekt Samael wurde sich mit der Klassifizierung von Bildern auseinander gesetzt. Grundlage für die Evaluierung der Arbeit ist der Datensatz der Caltech101. 
101 Kategorien
30-80 Samples pro Kategorie
Das Ziel ist es, mindestens zehn Prozent der Bilder korrekt zu klassifizieren.


Description of the problem (based on the sample images)
•Decision on features (justification, the measure itself)
•Classification details
•How the classifier was trained
–Training details
–Results (i.e. how good on training data, hints for subgroups, validity of features,…)
•Tests on independent test data
–Description of the test scenario
–Results and any conclusions from that
•Conclusions (how would you rate your approach)
\section{Einleitung}
\label{firstSec}

%\subsection{Softwarevalidation}
%\scshape State of the Art\\ 
\newline 
\normalfont

%------------------------------Literaturverzeichnis-----------------------------------------------------------------------
\newpage
\pagenumbering{Roman}	% römische Seitenzahlen
\setcounter{page}{6}
\renewcommand{\refname}{Referenzen}
\bibliographystyle{gerplain}
\addcontentsline{toc}{section}{Referenzen}
\bibliography{literatur}


\end{document}